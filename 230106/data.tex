%% samples.dtx  (with options: `authordraft')

%% For the copyright see the source file.

%% Any modified versions of this file must be renamed
%% with new filenames distinct from sample-authordraft.tex.

%% For distribution of the original source see the terms
%% for copying and modification in the file samples.dtx.
%% 
%% This generated file may be distributed as long as the
%% original source files, as listed above, are part of the
%% same distribution. (The sources need not necessarily be
%% in the same archive or directory.)
%%
%% Commands for TeXCount
%TC:macro \cite [option:text,text]
%TC:macro \citep [option:text,text]
%TC:macro \citet [option:text,text]
%TC:envir table 0 1
%TC:envir table* 0 1
%TC:envir tabular [ignore] word
%TC:envir displaymath 0 word
%TC:envir math 0 word
%TC:envir comment 0 0
%%
%%
%% The first command in your LaTeX source must be the \documentclass command.
\documentclass[sigconf,authordraft]{acmart}
%% NOTE that a single column version may required for 
%% submission and peer review. This can be done by changing
%% the \doucmentclass[...]{acmart} in this template to 
%% \documentclass[manuscript,screen]{acmart}
%% 
%% To ensure 100% compatibility, please check the white list of
%% approved LaTeX packages to be used with the Master Article Template at
%% https://www.acm.org/publications/taps/whitelist-of-latex-packages 
%% before creating your document. The white list page provides 
%% information on how to submit additional LaTeX packages for 
%% review and adoption.
%% Fonts used in the template cannot be substituted; margin 
%% adjustments are not allowed.

%%
%% \BibTeX command to typeset BibTeX logo in the docs
\AtBeginDocument{%
  \providecommand\BibTeX{{%
    \normalfont B\kern-0.5em{\scshape i\kern-0.25em b}\kern-0.8em\TeX}}}

%% Rights management information.  This information is sent to you
%% when you complete the rights form.  These commands have SAMPLE
%% values in them; it is your responsibility as an author to replace
%% the commands and values with those provided to you when you
%% complete the rights form.

 \setcopyright{acmcopyright}
 \copyrightyear{2018}
 \acmYear{2018}
 \acmDOI{XXXXXXX.XXXXXXX}

%% These commands are for a PROCEEDINGS abstract or paper.

\acmConference[Conference acronym 'XX]{Make sure to enter the correct
  conference title from your rights confirmation email}{June 03--05,
  2018}{Woodstock, NY}

%
%  Uncomment \acmBooktitle if th title of the proceedings is different
%  from ``Proceedings of ...''!
%
%\acmBooktitle{Woodstock '18: ACM Symposium on Neural Gaze Detection,
%  June 03--05, 2018, Woodstock, NY} 
\acmPrice{15.00}
\acmISBN{978-1-4503-XXXX-X/18/06}

\usepackage{indentfirst}
\usepackage{algorithm}
\usepackage{algpseudocode}
\usepackage{multirow}
\usepackage{subcaption}
%%
%% Submission ID.
%% Use this when submitting an article to a sponsored event. You'll
%% receive a unique submission ID from the organizers
%% of the event, and this ID should be used as the parameter to this command.
%%\acmSubmissionID{123-A56-BU3}

%%
%% For managing citations, it is recommended to use bibliography
%% files in BibTeX format.
%%
%% You can then either use BibTeX with the ACM-Reference-Format style,
%% or BibLaTeX with the acmnumeric or acmauthoryear sytles, that include
%% support for advanced citation of software artefact from the
%% biblatex-software package, also separately available on CTAN.
%%
%% Look at the sample-*-biblatex.tex files for templates showcasing
%% the biblatex styles.
%%

%%
%% For managing citations, it is recommended to use bibliography
%% files in BibTeX format.
%%
%% You can then either use BibTeX with the ACM-Reference-Format style,
%% or BibLaTeX with the acmnumeric or acmauthoryear sytles, that include
%% support for advanced citation of software artefact from the
%% biblatex-software package, also separately available on CTAN.
%%
%% Look at the sample-*-biblatex.tex files for templates showcasing
%% the biblatex styles.
%%

%%
%% The majority of ACM publications use numbered citations and
%% references.  The command \citestyle{authoryear} switches to the
%% "author year" style.
%%
%% If you are preparing content for an event
%% sponsored by ACM SIGGRAPH, you must use the "author year" style of
%% citations and references.
%% Uncommenting
%% the next command will enable that style.
%%\citestyle{acmauthoryear}

%%
%% end of the preamble, start of the body of the document source.
\begin{document}

%%
%% The "title" command has an optional parameter,
%% allowing the author to define a "short title" to be used in page headers.
\title{Mixed-Height-Cell Detailed Placement Algorithm for improving HPWL and minimizing displacement}

%%
%% The "author" command and its associated commands are used to define
%% the authors and their affiliations.
%% Of note is the shared affiliation of the first two authors, and the
%% "authornote" and "authornotemark" commands
%% used to denote shared contribution to the research.

\author{ }
\affiliation{%
  \institution{}
  \city{}
  \country{}
}

\author{}
\affiliation{%
  \institution{}
  \city{}
  \country{}
}

%%
%% By default, the full list of authors will be used in the page
%% headers. Often, this list is too long, and will overlap
%% other information printed in the page headers. This command allows
%% the author to define a more concise list
%% of authors' names for this purpose.

%%
%% The abstract is a short summary of the work to be presented in the
%% article.
\begin{abstract}
In modern VSLI design, mixed height standard cells have become more common as technology advances. Also, placement is one of the important steps in VLSI design flow and roughly consists of three stages: global placement, legalization and detailed placement. Global placement generates a rough placement solution that ignores overlaps between cells and alignment to rows. Legalization deals with global placement solution to legalize standard cells, i.e., to make cells overlap-free and aligned to rows while maintaining global placement quality by minimizing displacement. However, we also have to consider HPWL to improve routability at the following routing step. Detailed placement is for futher improvement of legalized placement solution such as HPWL reduction. In this work, we propose the detailed placement algorithm to improve HPWL while minimizing displacement to preserve the global placement performance from the mixed height standard cell legalization result. The experimental result based on ISPD2015 benchmarks shows that this algorithm achieves 10.05\% reduction in HPWL and 12.49\% reduction in displacement on average.
\end{abstract}

%% Keywords. The author(s) should pick words that accurately describe
%% the work being presented. Separate the keywords with commas.
\keywords{detailed placement, HPWL, displacement}

%% This command processes the author and affiliation and title
%% information and builds the first part of the formatted document.
\maketitle
\section{Introduction}

VLSI physical design is usually implemented with standard cell libraries for efficient and fast design. Traditionally standard cell libraries are composed of the same-height cells \cite{wang2004standard}. As technology advances, however, we began to adopt mixed height cells for efficient design that meets various needs such as high performance, low power and small area with the same functionality. \cite{baek2008ultra, dobre2015mixed}. Using mixed height cells means standard cell libraries have not only single-height cells but also multiple-height cells. For simple cells like inverters, we can use single-height cells. In contrast, for complex cells such as flip-flops, using multiple-height cells is more efficient in terms of power and performance. Therefore mixed-height cell design makes physical design more efficient but simultaneously makes it a complex problem. \par

Placement traditionally consists of three steps: global placement, legalization and detailed placement. Global placement tries to find out the optimal position of standard cells in order to optimize metrics such as wirelength, routability, etc., without considering overlaps between cells and alignment to row grids. Legalization removes overlaps between cells and makes them aligned to row grids. Its purpose is to minimize displacement from global placement position and ultimately maintain the performance of global placement as much as possible. Detailed placement, the last stage, refines the result of legalization to optimize some metrics concerning timing, power, wirelength, etc.  \par

There have been significant innovations on global placement \cite{kim2011simplr,he2013ripple,liu2013optimization,hsu2011routability,li2014analytical,kim2012complx} and legalization algorithm\cite{spindler2008abacus, zhu2018mixed, wang2017effective, hung2017mixed}, but not much on detailed placement algorithm. This is because detailed placement is affected by much more constraints than global placement and legalization. The work \cite{doll1994iterative} is a network flow-based detailed placement algorithm. It iteratively improves net lengths by using row-based cell placement. Many cells are placed simultaneouly by using network flow method. \cite{he2011ripple} is a congestion-driven detailed placement algorithm appropriate for maintaining congestion. \cite{monteiro2015analytical} is about detailed placement algorithm which targets on timing optimization such as reducing WNS and TNS violation. It uses effective skew optimization techniques based on swapping algorithm. Iterative cell swapping procedure continue greedily until the improvement stops. It has a limitation of runtime but is useful for timing optimization. \cite{diao2010effective} is an algorithm about optimizing HPWL for mixed-height standard cell. \cite{zhou2014effective} adopts a three step density aware detailed placement algorithm. With a legalization solution, iterative cell swapping procedure is first applied to optimize scaled-HPWL. After that, cell re-ordering is applied to reduce HPWL. Finally cell bloating and refinement are applied to balance bin utilization. This work pays much more attention to cell density reduction than to HPWL reduction.  \par

In this paper, we propose an effective detailed placement algorithm to improve HPWL while preserving global placement quality for mixed-height cell design. In order to improve total wirelength, we adopt a greedy iterative cell swapping algorithm with displacement constraints. \par
 The rest of our paper is organized as follows: Section 2 explains the motivation of this paper. Section 3 gives information on terminology used in our main algorithm. Section 4 explains the algorithm with an example in detail. Section 5 shows our experimental results and section 6 presents our conclusion.

\begin{table}
\caption{Legalization result for single height cell design and mixed height cell design}
\centering
\begin{tabular}{|c|c|cc|cc|}
\hline
\multirow{2}{*}{Benckmarks} & 
\multirow{2}{*}{Density} & 
\multicolumn{2}{c|}{Displacement} &
\multicolumn{2}{c|}{HPWL} \\
 &   & SC only & SC+DC & SC only & SC+DC     \\ 
\hline
des\_perf\_1    & 0.98   & 227550 & 681815 & 1281320 & 2022800  \\
\hline
fft\_1          & 0.84   & 62258  & 192497  & 283184 & 532551  \\
\hline
fft\_2          & 0.5    & 33613  & 38008   & 277016 & 286495   \\
\hline
matrix\_mult\_1 & 0.8    & 199207 & 242767  & 1504430 & 1572490  \\
\hline
fft\_a          & 0.25   & 28568  & 24057  & 661846 & 687357  \\
\hline
fft\_b          & 0.28   & 19714  & 21118  & 694253 & 698159   \\
\hline
edit\_dist\_a   & 0.46   & 89777  & 94949 & 4224540 & 4273450 \\
\hline
\end{tabular}
\end{table}

\section{Motivation}
As seen in Table 1, mixed-height cell legalization is much harder than single-height cell legalization. It is a result of legalization for single height cell design and mixed height cell design based on Cadence Innovus. We can see that almost every benchmark has increased displacement and HPWL after legalization for mixed height cases compared with single height cases. Especially for dense benchmarks like mgc\_des\_perf\_1 and mgc\_fft\_1, the amount of increase is much larger.
It is because as design gets dense, it is hard to meet constraints for legalization. 
Therefore we are focusing on improving displacement and HPWL after legalization. 
Our method is more effective when the design is dense.


\section{Preliminary and Terminology}
In this section, we explain some concepts that we defined to make our algorithm.
\subsection{HPWL-window, GP-Area, cellQueue}

\textit{HPWL-window} of a target net refers to a rectangular area bounded by HPWL of the target net. If the cells comprising the target net are moved toward inside of \textit{HPWL-window}, the target net’s HPWL gets reduced. \textit{GP-Area} is a diamond shaped area with its origin at global placement position. For any points in this area, the distance from its origin is less than or equal to penalized displacement. Penalized displacement is calculated by multiplying original displacement from global placement solution with penalty term, $\gamma$. The penalty term, $\gamma$ normally has a value between 0 and 1. \textit{GP-Area} is used to find candidate cells having smaller displacements than the original displacement. \textit{CellQueue} refers to a queue structure of cells connected to the target net. The cells are inserted to \textit{cellQueue} in the order of how adjacent to \textit{HPWL-window}’s boundary. This is because the more outer the cells are located, the more they can reduce \textit{HPWL-window} of the target net.


%However if we target on improving much more HPWL with sacrificing displacement, we can set $\gamma$ larger than 1.
%However, if our target is more on improving HPWL rather than displacement, 


\subsection{Cost function}
Our main objective is to minimize both HPWL and displacement. Therefore, we set our cost function as the linear combination of delta HPWL and delta displacement. It means that for every iteration, our cost minimizes HPWL and displacement greedily.
\begin{equation} 
    Cost = \alpha_1 * \Delta HPWL + \alpha_2 * \Delta displacement
\end{equation}

By adjusting $\alpha_1$ and $\alpha_2$, we can make a balance between two factors. 
\textit{Cost} should have a value less than or equal to 0 to monotonically decrease the total cost. 
We iteratively minimize this cost function for every net in our algorithm. 

\subsection{Power rail alignment}

In order to apply our algorithm to mixed height cell libraries, we should consider the power rail alignment before cell swapping. As you can see in Figure 1, odd height cells have both vdd and vss power rails on the end side. Thus when our algorithm swaps cells composed of odd height cells, we do not have to consider power rail alignment since flipping cells can easily solve this problem. However, even height cells have only one power rail, vdd or vss, so we consider power rail alignment. 

\begin{figure}[h]
  \centering
\begin{subfigure}{0.2\textwidth}
\includegraphics[width=\textwidth]{p1.png}
  \subcaption{odd height cell with both type of power rails.}
\end{subfigure}
% <— this is important. There should be no empty line here. 
\begin{subfigure}{0.2\textwidth}
\includegraphics[width=\textwidth]{p2.png}
  \subcaption{even height cell with one type of power rail.}
\end{subfigure}

\caption{power rail of different height standard cell.}
\end{figure}

\section{Our proposed algorithm}

In this section, we explain our detailed placement algorithm. First, we introduce the main algorithm flow. Then we describe how this algorithm works in detail with an example. \par
\begin{figure}[h]
  \centering
  \includegraphics[width=\linewidth]{1_flow.png}
  \caption{Our detailed placement algorithm flow}
\end{figure}

\begin{figure*}[h]
  \centering
\begin{subfigure}{0.33\textwidth}
\includegraphics[width=\textwidth]{1_swap.png}
  \subcaption{initial state.}
\end{subfigure}
% <— this is important. There should be no empty line here. 
\begin{subfigure}{0.33\textwidth}
\includegraphics[width=\textwidth]{2_swap.png}
  \subcaption{cell 9 is swapped with cell 5.}
\end{subfigure}
%
\begin{subfigure}{0.33\textwidth}
\includegraphics[width=\textwidth]{3_swap.png}
  \subcaption{cell 9 is swapped with cell 7.}
\end{subfigure}
\caption{The example of our algorithm including 3 nets.}
\end{figure*}



Our algorithm is shown in Algorithm 1. 
Line 1 constructs a priority queue of nets. 
The priority queue is prioritized by the size of HPWL because the more prioritized HPWL means that the net has more possibilities of improvements. 
building a prioriy queue take O(n) time, n refers to number of nets.
As you can see in Figure 3, every time we swap cell pairs in each \textit{while} loop, we update the priority queue for nets with modified HPWL. 
For updating priority queue, it needs only O(nlog(n)) time. %% n x logn
Line 2-19 describes the net by net optimizing procedure in the \textit{while} loop. 
After popping out the target net from the priority queue, we optimize both the total HPWL and displacement of cells connected to the target net by this swapping procedure. 
Line 3-7 generates \textit{HPWL-window}, \textit{cellQueue} and \textit{cellStatus} needed for the \textit{while} loop. \textit{HPWL-window} is generated especially to focus on the target net optimization. 
That is to say, when finding candidate cells for the swapping procedure, 
we do not consider cells located outside of \textit{HPWL-window}. This is because we mainly focus on optimizing the target net, even though the main objective is to optimize the total cost of all nets connected to the swapping cells. 
\textit{CellQueue} is composed of cells connected to the target net and they are inserted to the queue in order of adjacency to \textit{HPWL-window}'s boundary. 
\textit{CellStatus} is also composed of cells connected to the target net, and the initial status of each cell is assigned \textit{True}. The \textit{True} value means that the cell has a potential for improving the cost function. During the swapping procedure, if the cell is successfully swapped, its status gets the \textit{True} value. On the other hand, if it fails to be swapped, its status gets the \textit{False} value. 
The \textit{while} loop in line 8-17 iteratively swaps cells in \textit{cellQueue} to minimize the cost until no more improvement can be achieved. We dequeue the target cell out of the queue, do the swapping process, and enqueue the target cell to the queue again. There is a reason the popped cells are circularly inserted into the queue: even if the target cell does not succeed in finding a candidate cell that shows improvements, it can later find one after optimizing other cells. The \textit{while} loop continues until the \textit{cellStatus} of every cell becomes \textit{False}, meaning that no further improvement is possible. 

This inner \textit{while} loop takes O(m), m refers to a average number of sites in GP-Area. 
It is because we need to check \textit{GP-Area} to find candidate cells. 
Sites are rectangular area consists of unit width and unit height.
We implement a site map based on sites from which we can check whether cells are exists. 

\begin{equation} 
    m = \frac{1}{C} * \sum_{i=1}^{C} \frac{d_{i}^{2}}{siteX*siteY}
\end{equation}

C refers to the number of cells, 
$d_{i}$ refers to penalized original displacement of target cell i, 
siteX and siteY means unit width and unit height of standard cell respectively. 
If every cell is legalized well and their displacement are small, it takes small runtime. 
However, if cells are not legalized well and have larger displacement, it takes large runtime.

In line 18 the position of the moved cell in cellQueue is fixed. This means that the moved cells in \textit{cellQueue} will not be moved in the following procedure. This is because we assume that all the moved cells associated with the target net have found their optimal positions after the target net optimization procedure. This fixing process can also help to reduce the time complexity of this algorithm.
Therefore, overall time complexity is O(mn).
\newcommand{\factorial}{\ensuremath{\mbox{\sc Factorial}}}
\begin{algorithm} 
\caption{Mixed-Cell Detailed Placement Algorithm}\label{euclid}
\algrenewcommand\algorithmicrequire{\textbf{Input:}}
\algrenewcommand\algorithmicensure{\textbf{Output:}}
\begin{algorithmic}[1]
\Require Legalization result, Global placement result
\Ensure Detailed placement result
\State Construct priority queue \textit{pq} of nets with priority of |HPWL| in non-increasing order.  
\While{\textit{pq} is not empty}
\State $n_i = pq.dequeue()$
\State $HPWLwindow \gets getWindowHPWL(n_i)$
\State $cellQueue \gets getCellQueue(n_i)$
\State $cellStatus \gets getCellStatus(cellQueue, true)$
\State $Create\:a\:empty\:set\:movedCell$
   \While{$ Not\:every\:cellStatus\:is\:false$}
      \State $targetCell\gets cellQueue.dequeue()$
      \State $candidateCells\gets getCandidate(targetCell, HPWLwindow)$
      \If {$candidateCells $}
      \State $swapCell\gets calculateCost(candidateCells)$
      \State $swap\:targetCell\: with \:swapCell$
      \State $insert\: targetCell \:to \:movedCell$
      \EndIf
      \State $cellQueue.enqueue(targetCell)$
   \EndWhile\label{euclidendwhile}
\State $Fix\:position\:of\:movedCell$
\EndWhile\label{euclidendwhile}
\end{algorithmic}
\end{algorithm}

 More specifically, we will explain our algorithm with a simple example shown in Figure 2. 
We have three nets \{net1, net2, net3\} with HPWL of \{100, 50, 60\}. net1 is connected to group1 = \{cell1, cell3, cell4, cell5, cell8, cell9\}, net2 is connected to group2 = \{cell2, cell5, cell6, cell10\} and net3 is connected to group3= \{cell7, cell8, cell11, cell12\}. We first construct a priority queue of nets \textit{pq} = \{net1, net3, net2\} prioritized by the size of HPWL. After that, we pop out net1 from \textit{pq} as our target net and construct \textit{cellQueue} and \textit{cellStatus} of net1. We push cells into the \textit{cellQueue} in the order of adjacency to \textit{HPWL-window}’s boundary, which results in \textit{cellQueue} = \{cell9, cell1, cell4, cell8, cell5, cell3\} and \textit{cellsStatus} = \{true, true, true, true, true, true\}. Then we will pop out cell9 from \textit{cellQueue} and find candidate cells. To find candidate cells, we draw \textit{GP-Area} with its origin at global placement position of cell9. We get candidate cells from the intersection of \textit{GP-Area} and \textit{HPWL-window}, which results in \{cell5, cell6, cell7\}. After this, we calculate each cost of swapping cell9, the target cell, with \{cell5, cell6, cell7\}, the candidate cells. In Figure 2(b), we can see that if cell9 is swapped with cell5, it results in the decrease in HPWL of net2 and the displacement of cell9 while maintaining HPWL of net1. In the meantime, in Figure 2(c), if cell9 is swapped with cell7, it results in the decrease of HPWL of net 1 and 3 and the displacement of cell9. Let’s assume that swapping cell9 with cell7 results in the lowest cost. We swap cell9 with cell7 and update \textit{cellQueue} and \textit{cellStatus} as \textit{cellQueue} = \{cell1, cell4, cell8, cell5, cell3, cell9\} and \textit{cellStatus} = \{true, true, true, true, true, true\}. Then we go through the same swapping process with cell1, the target cell. If we fail to find candidate cells of cell1, we get \textit{cellQueue} = \{cell4, cell8, cell5, cell3, cell9, cell1\} and \textit{cellStatus} = \{true, true, true, true, true, false\}. We go through the same procedures repeatedly until every \textit{cellStatus} becomes \textit{False}. Then the \textit{while} loop ends finally and \textit{pq} is updated. The \textit{pq} is updated as \{net2, net3\} because HPWL of net3 is decreased from 60 to 45. We continue this procedure until all nets are traversed, i.e. \textit{pq} is empty. 

\begin{figure}[h]
  \centering
  \includegraphics[width=\linewidth]{1_pq.png}
  \caption{priority queue of net priortized by the size of HPWL}
\end{figure}
\section{Experimental result}

\begin{table*}
\caption{Experimental result of our detailed placement algorithm}
\centering
\begin{tabular}{|c|c|c|c|c|c|ccc|}
\hline
\multirow{2}{*}{Benckmarks} & \multirow{2}{*}{\#DC} & \multirow{2}{*}{\#SC} & 
\multirow{2}{*}{Density} & \multirow{2}{*}{Displacement(um)} & \multirow{2}{*}{HPWL(um)} & \multicolumn{3}{c|}{Our DP result}  \\
             &&&     &    &              & Displacement(\%)   & HPWL(\%) & runtime(s)    \\ 
\hline
mgc\_des\_perf\_1    & 8802   & 103842 & 0.91 & 681815 & 2022800 & 574819(15.69\%) & 1489990(26.34\%) & 586  \\
\hline
mgc\_des\_perf\_1    & 12354  & 100290 & 0.91 & 1054170 & 2651610 & 843210(20.01\%) & 1730880(34.72\%) & 1172  \\
\hline
mgc\_fft\_1          & 2708   & 29573 & 0.84  & 192497 & 532551 & 137076(28.79\%) & 369260(30.55\%) & 239  \\
\hline
mgc\_fft\_1          & 3932   & 28349 & 0.84  & 175038 & 503853 & 124364(28.95\%) & 355933(29.35\%) & 171  \\
\hline
mgc\_fft\_2          & 5198   & 27083 & 0.5   & 38008  & 286495 & 37936(0.23\%)  & 285867(0.17\%) & 2  \\
\hline
mgc\_matrix\_mult\_1 & 18359  & 136966 & 0.8  & 242767 & 1572490  & 236664(2.51\%)  & 1550670(1.39\%) & 12  \\
\hline
mgc\_matrix\_mult\_1 & 27512  & 127813 & 0.8  & 275105 & 1616840 & 263467(4.23\%)  & 1581010(2.22\%) & 12 \\
\hline
mgc\_fft\_a          & 5332   & 25299 & 0.25  & 24057  & 687357 & 24054(0.01\%)  & 687338(0.003\%) & 1  \\
\hline
mgc\_fft\_b          & 1907   & 28724 & 0.28  & 21118  & 698159 & 21113(0.02\%)  & 698136(0.003\%) & 1  \\
\hline
mgc\_edit\_dist\_a   & 18576  & 108843 & 0.46 & 94949  & 4273450 & 94865(0.09\%)  & 4272970(0.01\%)  & 6  \\
\hline
average   & -  & - & - & - & - & 10.05\%  & 12.49\%  & 240  \\
\hline
\end{tabular}
\end{table*}

 The proposed mixed-height-cell detailed placement algorithm was implemented in the C++ programming language. We used the 2015 ISPD Detailed-Routing-Driven Placement Contest benchmarks \cite{bustany2015ispd}. And we modified the benchmarks by doubling the height of some proportion of cells and keeping their area the same since the given benchmarks only provide single-height standard cell libraries. Our experiments were performed with Intel-i7 CPU at 4.7GHz and 64GB memory. The global placement result and the legalization result were acquired from the state-of-the-art commercial tool Cadence Innovus. 
 
 Table 2 shows the information of our benchmarks and our detailed placement results. \#DC means the number of double-height cells, and \#SC means the number of single-height cells. Displacement and HPWL are the result of legalization acquired by Innovus tool. The last three columns, Our DP result, show the improved displacement, HPWL and runtime. The experimental result shows that on average it has about 10.05\% gain in HPWL and 12.49\% gain in displacement on benchmarks. 
 
 For benchmarks with enough room to improve with their relatively high average displacement, it shows great improvements in both HPWL and displacement. For three benchmarks; mgc\_des\_perf\_1, mgc\_fft\_1 and mgc\_matrix\_mult\_1, we experimented twice, the second time increasing the proportion of \#DC. Except mgc\_fft\_1, it gives increased displacement after legalization. Especially for mgc\_des\_perf\_1 benchmark which has the highest density, the displacement increased about 54\%. It shows 4.32\% gain in displacement and 8.38\% gain in HPWL compared with the first experiment of the same benchmark. For benchmarks with lower density and small average displacement, there is not much improvement in HPWL and displacement when our alogrithm is applied. Instead it shows a very short runtime.   

\section{Conclusion}

We presented a new effective detailed placement algorithm. It is a cell swapping based algorithm and can be applicable to mixed-height cell libraries. It optimizes not only HPWL but also displacement from the legalization result, which means it keeps global placement performance as much as possible. Therefore, additional detailed placement algorithms can be applied after this algorithm without any loss. It shows 10.05\% improvement in HPWL and 12.49\% in displacement on average in 2015 ISPD Detailed-Routing-Driven Placement Contest benchmarks. This algorithm is much more effective when design gets complex and dense. Complex and dense design makes legalization harder which results in larger displacement, meaning it has a large potential for improvements when this algorithm is applied. If the design is not complex and has small displacement, it does not have a higher potential for improvements, but it can be done in a very short time. 

%%
%% The next two lines define the bibliography style to be used, and
%% the bibliography file.
\bibliographystyle{ACM-Reference-Format}
\bibliography{sample-base}

\end{document}
\endinput

%% End of file `sample-authordraft.tex'.
